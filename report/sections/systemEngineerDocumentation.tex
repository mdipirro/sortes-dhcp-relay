\section{Documentation for system engineers}
%\subsection{Compilation and download}
%The program comes with a \texttt{Makefile} that can be used to compile the program. To this end, the right command to use is \texttt{make timer}. It will generate some files in the current directory (\texttt{.}) and in \texttt{./Objects/}. In order to download the program into the naked computer, the user should follow the following steps (supposing the router is correctly configured):
%\begin{itemize}
%	\item Run \texttt{tftp 192.168.97.60} in the same directory as \texttt{timer.c}. The \texttt{tftp} environment will start;
%	\item \texttt{binary}, to send the program as binary;
%	\item \texttt{trace}, to see what happens;
%	\item \texttt{verbose}, to see more information;
%	\item \texttt{put timer.hex}.
%\end{itemize}
%Please notice that the last command has to be run only when the board is ready to receive the program. This happens during the three seconds after its reboot.

\subsection{Debug mode}
If something is not working properly a special mode can be activated which allows to view information messages during critical phases of the program execution. The \textbf{UART interface} is used to transmit the debug messages to an external terminal (typically a PC); the receiver must be connected to the RS232 port of the Olimex board and it must be set with these parameters:
\begin{itemize}
	\item Baud rate: 9600
	\item Data bits: 8
	\item Stop bits: 1
	\item Parity: Odd
\end{itemize}
The debug mode is \textbf{disabled by default}; it can be activated adding the option \texttt{-DUART\_DEBUG\_ON} to the CFLAGS in the Makefile and recompiling the code. Some predefined macros are used to print single characters (\texttt{DEBUGCHAR}), blocks of characters of any lenght (\texttt{DEBUGBLOCK}) and strings (\texttt{DEBUGMSG}) throughout the code; these macros are only effective in debug mode.