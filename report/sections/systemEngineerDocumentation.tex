\section{Documentation for system engineers}
\subsection{Compilation and download}
The program comes with a \texttt{Makefile} that can be used to compile the program. To this end, the right command to use is \texttt{make}. It will generate some files in the current directory (\texttt{.}) and in \texttt{./Objects/}. In order to download the program into the naked computer, the user should follow the following steps (supposing the router is correctly configured):
\begin{itemize}
	\item Run \texttt{tftp 192.168.97.60} in the same directory as \texttt{DHCPRelay.c}. The \texttt{tftp} environment will start;
	\item \texttt{binary}, to send the program as binary;
	\item \texttt{trace}, to see what happens;
	\item \texttt{verbose}, to see more information;
	\item \texttt{put DHCPRelay.hex}.
\end{itemize}
The last command has to be run only when the board is ready to receive the program. This happens during the three seconds after its reboot.

The board comes with a RouterBoard. It is configured with two different networks in order to test the relay. A LAN (\textbf{L}ocal \textbf{A}rea \textbf{N}etwork) comprises Ports 2, 3, 4, and 5; a WAN (\textbf{W}ide \textbf{A}rea \textbf{N}etwork) is set at Port 1. The DHCP server should be connected to the WAN, while both relay and clients should be on Ports from 2 to 5. The router built-in DHCP server should be disabled in order to let the relay work properly. Once disabled, it is not possible to communicate neither with the router itself nor with the board. Every device should be assigned a static IP address for communication purposes (meaning configuring the router and sending the program to the board). Examples are as follows:
\begin{itemize}
	\item 192.168.97.15 to send to program to the board;
	\item 192.168.88.10 to configure the router.
\end{itemize}

\subsection{DHCP server configuration}
A DHCP server must be added to the network and configured to assign a free IP address within a certain range to the clients in the same subnet of the relay.
On UNIX-based systems, the \textit{Internet Systems Consortium DHCP Server} (\texttt{dhcpd}) can be used as a daemon which provides this service; Listing~\ref{lst:dhcpd} shows a sample configuration using the addresses specified in the project.\\
\vbox{
	\lstinputlisting[label=lst:dhcpd,style=CStyle,caption={Sample dhcpd.conf}]{../src/dhcpd.conf}
}

\subsection{Debug mode}
If something is not working properly a special mode can be activated which allows to view information messages during critical phases of the program execution. The \textbf{UART interface} (\textbf{U}niversal \textbf{A}synchronous \textbf{R}eceiver-\textbf{T}ransmitter) is used to transmit the debug messages to an external terminal (typically a PC); the receiver must be connected to the RS232 port of the Olimex board and it must be set with these parameters:
\begin{itemize}
	\item Baud rate: 9600
	\item Data bits: 8
	\item Stop bits: 1
	\item Parity: Odd
\end{itemize}
The debug mode is \textbf{disabled by default}; it can be activated adding the option \texttt{-DUART\_DEBUG\_ON} to the CFLAGS in the Makefile and recompiling the code. Some predefined macros are used to print single characters (\texttt{DEBUGCHAR}), blocks of characters of any length (\texttt{DEBUGBLOCK}) and strings (\texttt{DEBUGMSG}) throughout the code; these macros are only effective in debug mode.

Please note that this debug mode only works in a ``blocking'' mode, and may therefore result in additional delays while executing the software.